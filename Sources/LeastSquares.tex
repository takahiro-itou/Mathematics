%%  -*-  coding:utf-8  -*-
\documentclass[12pt]{jsarticle}

%%  -*-  coding:utf-8  -*-

%%
%%  プリアンブル
%%
\usepackage{amsthm}
\usepackage{amsfonts}
\usepackage{amsmath}

\theoremstyle{definition}
\newtheorem{definition}{定義}[section]
\newtheorem{lemma}{補題}[section]
\newtheorem{theorem}{定理}[section]
\newtheorem{proposition}{命題}[section]
\newtheorem{problem}{問題}[section]
\newtheorem{example}{例}[section]


\def\error{\varepsilon}
\def\sumdata{\sum_{i=1}^{N}}

\title{最小二乗法}

\begin{document}

\maketitle
\tableofcontents

\section{概要}
ある観測データの組 $(x, y)$ は、モデル関数 $\function{f}{x}$ と
誤差 $\varepsilon$ の和で
\begin{equation}
  y = \function{f}{x} + \error
\end{equation}
と表せるとする。

$N$ 組の観測データ $(x_i, y_i)$ が与えられたとき、
各 $x_i$ についてモデルから得られる予測値
$\function{f}{x_i}$ と、実際の測定値 $y_i$ の誤差 $\error_i$ つまり
\begin{equation}
  \error_i := \left| \function{f}{x_i} - y_i \right|
\end{equation}
の和を最小にすることを考えたい。
しかし、絶対値があると後々の計算が大変なので、
実際には、誤差の二乗和

\begin{equation}
  \Delta := \frac{1}{2} \sumdata \error_i^2 = \frac{1}{2} \sumdata
  \left( \function{f}{x_i} - y_i \right)^2 \label{eq:least_square}
\end{equation}

を最小にするようなモデル $\function{f}{x}$ を求める。
なお、係数 $\frac{1}{2}$ は、後の計算の時に、
多少計算が楽になるのを見越して付けているだけで、本質ではない。

以下のセクションでは、簡単なものから複雑なものまで、
具体的に求めてみる。

\section{線形回帰 (直線) の場合}

モデル関数は
\begin{equation}
  f \left( x \right) = a x + b
\end{equation}
であるから、式 (\ref{eq:least_square}) は
\begin{equation}
  \Delta = \frac{1}{2} \sumdata \error_i = \frac{1}{2}
  \sumdata \left( a x_i + b - y_i \right) ^2
\end{equation}
となる。これを最小化する $a, b$ を求めればよい。
これを $a, b$ に関する二次関数と見れば、この関数が最小値を取る点では
$a$ および $b$ についての偏微分がゼロであるから、

\begin{eqnarray*}
  \partd{}{a} \Delta &=& 0, \\
  \partd{}{b} \Delta &=& 0
\end{eqnarray*}
を連立して解けばよい。実際に計算してみると

\begin{eqnarray*}
  \partd{}{a} \Delta = \sumdata \left(
  a x_i + b - y_i \right) x_i
  = a \sumdata x_i^2 +b \sumdata x_i - \sumdata x_i y_i
  &=& 0, \\
  \partd{}{b} \Delta = \sumdata \left(
  a x_i + b - y_i \right) 1
  = a \sumdata x_i + b \sumdata 1 - \sumdata y_i, &=& 0,
\end{eqnarray*}

すなわち

\begin{equation}
  \left\{
  \begin{array}{ll}
    \left( \sumdata x_i^2 \right) a + \left( \sumdata x_i \right) b
    &= \sumdata x_i y_i, \\
    \left( \sumdata x_i \right) a + \left( \sumdata 1 \right) b
    &= \sumdata y_i,
  \end{array}
  \right.
\end{equation}

を解けばよい。行列を使って書けば

\begin{equation}
  \begin{bmatrix}
    \sumdata x_i^2, & \sumdata x_i \\
    \sumdata x_i,   & \sumdata 1
  \end{bmatrix}
  \begin{bmatrix}
    a \\
    b
  \end{bmatrix}
  =
  \begin{bmatrix}
    \sumdata x_i y_i \\
    \sumdata y_i
  \end{bmatrix}
\end{equation}

よって、
\begin{equation}
  Det := \left( \sumdata x_i^2 \right) \left( \sumdata 1 \right)
  - \left( \sumdata x_i \right)^2
\end{equation}
とおけば、

\begin{eqnarray}
  \begin{bmatrix}
    a \\
    b
  \end{bmatrix}
  &=&
  \begin{bmatrix}
    \sumdata x_i^2, & \sumdata x_i \\
    \sumdata x_i,    & \sumdata 1
  \end{bmatrix} ^{-1}
  \begin{bmatrix}
    \sumdata x_i y_i \\
    \sumdata y_i
  \end{bmatrix} \nonumber \\
  &=&
  \frac{1}{Det}
  \begin{bmatrix}
    \sumdata 1,     & - \sumdata x_i, \\
    - \sumdata x_i, & \sumdata x_i^2
  \end{bmatrix}
  \begin{bmatrix}
    \sumdata x_i y_i \\
    \sumdata y_i
  \end{bmatrix}
\end{eqnarray}

さらに $\sumdata 1 = N$ であるから、最終的に

\begin{eqnarray}
  a &=& \frac{N \sumdata x_i y_i -
    \left( \sumdata x_i \right) \left( \sumdata y_i \right)
  }{N \sumdata x_i^2 - \left( \sumdata x_i \right)^2}, \nonumber \\
  b &=& \frac{
    \left( \sumdata x_i^2 \right) \left( \sumdata y_i \right)
    -
    \left( \sumdata x_i y_i \right) \left( \sumdata x_i \right)
  }{N \sumdata x_i^2 - \left( \sumdata x_i \right)^2}, \nonumber \\
\end{eqnarray}

\section{一般的な多項式の場合}

モデル関数は
\begin{equation}
  f \left( x \right) = a_0 + a_1 x^1 + \cdots + a_{n-1} x^{n-1} + a_n x^n
\end{equation}

であるから、式 (\ref{eq:least_square}) は

\begin{equation}
  \Delta = \frac{1}{2}
  \sumdata \left( a_0 + a_1 x_i + + a_2 x_i^2
  + \cdots + a_{n-1} x_i^{n-1} + a_n x_i^n - y_i \right) ^2
\end{equation}
となる。先ほどと同様に $a_0, a_a, \ldots, a_n$ で偏微分すると

\begin{eqnarray*}
  \partd{}{a_0} \Delta &=& \sumdata \left(
  a_0 + a_1 x_i + \cdots a_n x_i^n  - y_i  \right) 1, \nonumber \\
  \partd{}{a_1} \Delta &=& \sumdata \left(
  a_0 + a_1 x_i + \cdots a_n x_i^n  - y_i  \right) x_i, \nonumber \\
  \partd{}{a_2} \Delta &=& \sumdata \left(
  a_0 + a_1 x_i + \cdots a_n x_i^n  - y_i  \right) x_i^2, \nonumber \\
  \cdots \nonumber \\
  \partd{}{a_n} \Delta &=& \sumdata \left(
  a_0 + a_1 x_i + \cdots a_n x_i^n  - y_i  \right) x_n^2, \nonumber \\
\end{eqnarray*}

この各式がゼロとおいて、移項して行列の形に纏めると

\begin{equation}
  \begin{bmatrix}
    \sumdata x_i^0, & \sumdata x_i^1, & \sumdata x_i^2,
    & \cdots, & \sumdata x_i^n \\
    \sumdata x_i^1, & \sumdata x_i^2, & \sumdata x_i^3,
    & \cdots, & \sumdata x_i^{n+1} \\
    \sumdata x_i^2, & \sumdata x_i^3, & \sumdata x_i^4,
    & \cdots, & \sumdata x_i^{n+2} \\
    \vdots, & \vdots, & \vdots, & \ddots, & \vdots, \\
    \sumdata x_i^n, & \sumdata x_i^{n+1} & \sumdata x_i^{n+2},
    & \cdots, & \sumdata x_i^{2n}
  \end{bmatrix}
  \begin{bmatrix}
    a_0 \\ a_1 \\ a_2 \\ \vdots \\ a_n
  \end{bmatrix}
  =
  \begin{bmatrix}
    \sumdata x_i^0 y_i \\
    \sumdata x_i^1 y_i \\
    \sumdata x_i^2 y_i \\
    \vdots \\
    \sumdata x_i^n y_i
  \end{bmatrix}
\end{equation}

後はこれに掃き出し法や LR 分解など、
連立一次方程式の解法アルゴリズムを適用すれば良い。

\section{発展編}

\subsection{円の場合}

円の場合は $y = \function{f}{x}$ の形で
書けない (書けなくはないが面倒) ので工夫が必要である。
中心 $(a, b)$ 半径 $r$ の円の場合 $(x, y)$ は
\begin{equation}
  (x - a)^2 + (y - b)^2 = r^2
\end{equation}
を満たす。

今までの方法では$y = \function{f}{x}$ という形式だったが、
これを変形すると 点 $(x_i, y_i)$ がこのモデル関数のグラフ上に乗っているならば、
\begin{equation}
  \function{f}{x_i} - y_i = 0
\end{equation}
を満たすべきであって、$\error_i$ は、
この左辺の値 (と $0$ との誤差) である、
とも言える。これと同様に考えると、

\begin{equation}
  \error_i = (x_i - a)^2 + (y - b)^2 - r^2
\end{equation}
で同様の手続きをすればよいが、そのままでは上の式を
二乗したときに $a$, $b$, $r$ に関する四次式になってしまうので、
最小値を求めることが難しい (偏微分がゼロの点が最小値とは限らない) 。
そこで、以下のように式変形を行う。

\begin{eqnarray}
  \error_i &=& (x_i - a)^2 + (y_i - b)^2 - r^2 \nonumber \\
  &=& x_i^2 - 2 a x_i + a^2 + y_i^2 - 2 b y_i + b^2 - r^2 \nonumber \\
  &=& - 2 a x_i - 2 b y_i + (a^2 + b^2 - r^2) + x_i^2 + y_i^2 \nonumber \\
  &=& A x_i + B y_i + C + x_i^2 + y_i^2
\end{eqnarray}

ただし
\begin{eqnarray}
  A &=& -2a, \nonumber \\
  B &=& -2b, \nonumber \\
  C &=& a^2 + b^2 - r^2 \label{eq:lsm_circle_assign}
\end{eqnarray}

である。すると式 (\ref{eq:least_square}) は

\begin{equation}
  \Delta := \frac{1}{2} \sumdata \error_i^2
  = \frac{1}{2} \sumdata \left(
  A x_i + B y_i + C + x_i^2 + y_i^2
  \right)^2
\end{equation}
となって $A, B, C$ の二次式となる。
$x_i, y_i$ は測定データで既知つまり定数であるから、
これが四乗になっていても問題ないことに注意する。

これを $A, B, C$ で偏微分すると

\begin{eqnarray}
  \partd{\Delta}{A} &=&
  \sumdata \left(A x_i + B y_i + C + x_i^2 + y_i^2 \right)
  x_i = 0 \nonumber \\
  \partd{\Delta}{B} &=&
  \sumdata \left(A x_i + B y_i + C + x_i^2 + y_i^2 \right)
  y_i = 0 \nonumber \\
  \partd{\Delta}{C} &=&
  \sumdata \left(A x_i + B y_i + C + x_i^2 + y_i^2 \right)
  1 = 0
\end{eqnarray}

従って

\begin{equation}
  \begin{bmatrix}
    \sumdata x_i x_i, & \sumdata x_i y_i, & \sumdata x_i \\
    \sumdata x_i y_i, & \sumdata y_i y_i, & \sumdata y_i \\
    \sumdata x_i,     & \sumdata y_i,     & \sumdata 1
  \end{bmatrix}
  \begin{bmatrix}
    A \\ B \\ C
  \end{bmatrix}
  = -
  \begin{bmatrix}
    \sumdata \left( x_i^2 + y_i^2 \right) x_i \\
    \sumdata \left( x_i^2 + y_i^2 \right) y_i \\
    \sumdata \left( x_i^2 + y_i^2 \right) 1
  \end{bmatrix}
\end{equation}

後は、この連立方程式を解いて $A, B, C$ を決定し、
式 (\ref{eq:lsm_circle_assign}) に代入して

\begin{eqnarray}
  a &=& - \frac{A}{2}, \\
  b &=& - \frac{B}{2}, \\
  r &=& \sqrt{a^2 + b^2 - C} = \frac{\sqrt{A^2 + B^2 - 4C}}{2}
\end{eqnarray}

が得られる。

\subsection{球の場合}

球の場合は、変数が増えるだけで、円の場合と全く同じである。

中心 $(a, b, c)$ 半径 $r$ の球の場合、
点 $(x_i, y_i, z_i)$ がこの球の表面上にある時

\begin{equation}
  (x_i - a)^2 + (y_i - b)^2 + (z_i - c) - r^2 = 0
\end{equation}

を満たす筈である。よって誤差 $\error_i$ は

\begin{eqnarray}
  \error_i &=& (x_i - a)^2 + (y_i - b)^2 + (z_i - c)^2 - r^2 \nonumber \\
  &=& x_i^2 - 2 a x_i + a^2  + y_i^2 - 2 b y_i + b^2
  + z_i^2 - 2 c z_i + c^2  - r^2 \nonumber \\
  &=& - 2 a x_i - 2 b y_i - 2 c z_i
  + (a^2 + b^2 + c^2 - r^2) + x_i^2 + y_i^2 +z_i^2 \nonumber \\
  &=& A x_i + B y_i + C z_i + D + x_i^2 + y_i^2 + z_i^2
\end{eqnarray}

となる。ただし
\begin{eqnarray}
  A &=& -2a, \nonumber \\
  B &=& -2b, \nonumber \\
  C &=& -2c, \nonumber \\
  D &=& a^2 + b^2 + c^2 - r^2 \label{eq:lsm_ball_assign}
\end{eqnarray}

である。すると式 (\ref{eq:least_square}) は

\begin{equation}
  \Delta := \frac{1}{2} \sumdata \error_i^2
  = \frac{1}{2} \sumdata \left(
  A x_i + B y_i + C z_i + D + x_i^2 + y_i^2 + z_i^2
  \right)^2
\end{equation}

となって、これを $A, B, C, D$ で偏微分すると

\begin{eqnarray}
  \partd{\Delta}{A} &=&
  \sumdata \left(A x_i + B y_i + C z_i + D + x_i^2 + y_i^2 + z_i^2 \right)
  x_i = 0 \nonumber \\
  \partd{\Delta}{B} &=&
  \sumdata \left(A x_i + B y_i + C z_i + D + x_i^2 + y_i^2 + z_i^2 \right)
  y_i = 0 \nonumber \\
  \partd{\Delta}{C} &=&
  \sumdata \left(A x_i + B y_i + C z_i + D + x_i^2 + y_i^2 + z_i^2 \right)
  z_i = 0 \nonumber \\
  \partd{\Delta}{D} &=&
  \sumdata \left(A x_i + B y_i + C z_i + D + x_i^2 + y_i^2 + z_i^2 \right)
  1 = 0
\end{eqnarray}

従って

\begin{equation}
  \begin{bmatrix}
    \sumdata x_i x_i, & \sumdata y_i x_i, & \sumdata z_i x_i, & \sumdata x_i \\
    \sumdata x_i y_i, & \sumdata y_i y_i, & \sumdata z_i y_i, & \sumdata y_i \\
    \sumdata x_i z_i, & \sumdata y_i z_i, & \sumdata z_i z_i, & \sumdata z_i \\
    \sumdata x_i,     & \sumdata y_i,     & \sumdata z_i,     & \sumdata 1
  \end{bmatrix}
  \begin{bmatrix}
    A \\ B \\ C \\ D
  \end{bmatrix}
  = -
  \begin{bmatrix}
    \sumdata \left( x_i^2 + y_i^2 + z_i^2 \right) x_i \\
    \sumdata \left( x_i^2 + y_i^2 + z_i^2 \right) y_i \\
    \sumdata \left( x_i^2 + y_i^2 + z_i^2 \right) z_i \\
    \sumdata \left( x_i^2 + y_i^2 + z_i^2 \right) 1
  \end{bmatrix}
\end{equation}

後は、この連立方程式を解いて $A, B, C$ を決定し、
式 (\ref{eq:lsm_ball_assign}) に代入して

\begin{eqnarray}
  a &=& - \frac{A}{2}, \\
  b &=& - \frac{B}{2}, \\
  c &=& - \frac{C}{2}, \\
  r &=& \sqrt{a^2 + b^2 + c^2- D} = \frac{\sqrt{A^2 + B^2 + C^2 - 4D}}{2}
\end{eqnarray}

が得られる。

\section{(応用) 疑似逆行列}

行列 $A$ が正方行列であって、逆行列 $A^{-1}$ が存在する時、
連立方程式 $A \vector{x} = \vector{b}$ は、
$\vector{x} = A^{-1} \vector{b}$
と解くことができる。

ここでは $A$ が正方行列でない場合を考える。
今回は、行の方が多いつまり $A$ が $m \times n$ 行列の時に
$m > n$ である場合だけ考えよう。
この場合、連立方程式で言うならば、変数の数 $n$ よりも
式の数 $m$ の方が多いのだから、一般には解が存在しない。
そこで、近似解を求めることで妥協する。

まず、連立方程式の右辺を移項して、右辺をゼロに変形しておく。
\begin{equation}
  A \vector{x} - \vector{b} = \vector{0}
\end{equation}
このような $x$ は存在しないから、先ほどまでの最小二乗法と同じ考えで、
誤差に相当する
\begin{equation}
  \| A \vector{x} - \vector{b} \| \label{eq:error_norm}
\end{equation}
が最小になるように $\vector{x}$ を求めることにする。

$m \times n$ 行列 A に対して、以下の 4 条件を満たす $n \times m$ 行列
$A^{+}$ がただ一つ定まる :
\begin{eqnarray}
  A A^{+} A = A, \\
  A^{+} A A^{+} = A^{+} \\
  (A A^{+})^{*} = A A^{+} \\
  (A^{+} A)^{*} = A^{+} A
\end{eqnarray}
なお、$A^{*}$ は $A$ の
随伴行列 (各要素の複素共役を取りさらに転置した行列) を表す。

この行列 $A^{+}$ を $A$ の疑似逆行列と呼ぶ。
$A$ が正則でなくても $A^{+}$ は定まる。
$A$ が正則ならば $A$ の逆行列はこの性質を満たす (つまり $A^{+} = A^{-1}$) 。

詳しくは省略するが $A$ の各列が一時独立であれば、
$A^{*} A$ は可逆であり、
\begin{equation}
  A^{+} = \left( A^{*} A \right)^{-1} A^{*}
\end{equation}
が成り立つ。この時 存在しない $A^{-1}$ の代わりに $A^{+}$ を用いることで
\begin{equation}
  \vector{x} = A^{+} \vector{b}
\end{equation}
とすれば、この $\vector{x}$ は 式 (\ref{eq:error_norm}) を
最小にするベクトルである。

\end{document}
