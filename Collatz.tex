%%  -*-  coding:utf-8  -*-

%%
%%
%%
\section{コラッツ=角谷 予想}

\begin{definition}
$\forall m \in \mathbb{N}$ に対して、
関数 $f(m)$ を
\begin{equation}
f(m) =
\begin{cases}
m / 2  & \text{if} \quad (m \equiv 0) \mod 2 \\
3m + 1 & \text{if} \quad (m \equiv 1) \mod 2
\end{cases}
\end{equation}
とする。
さらに、数列 $a_n$ を
\begin{eqnarray}
a_0 &=& m \\
a_i &=& f(a_{i-1}) \quad \text{for~} i \ge 1
\end{eqnarray}
で定義する。
\end{definition}

\begin{example}
$m=12$ の時、
\begin{align}
a_0 = 12, && a_1 = 6, && a_2 = 3, && a_3 = 10, && a_4 = 5, \nonumber \\
a_5 = 16, && a_6 = 8, && a_7 = 4, && a_8 = 2,  && a_9 = 1, \nonumber \\
a_{10} = 4, && a_{11} = 2, && a_{12} = 1, && \ldots
\end{align}
となる。
\end{example}

\begin{problem}
どんな初期値 $m \in \mathbb{N}$ から始めても
数列 $a_n$ は必ず $1$ を含むか?
\end{problem}
