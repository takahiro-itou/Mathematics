%%  -*-  coding:utf-8  -*-
\documentclass[12pt]{jsarticle}

%%  -*-  coding:utf-8  -*-

%%
%%  プリアンブル
%%
\usepackage{amsthm}
\usepackage{amsfonts}
\usepackage{amsmath}

\theoremstyle{definition}
\newtheorem{definition}{定義}[section]
\newtheorem{lemma}{補題}[section]
\newtheorem{theorem}{定理}[section]
\newtheorem{proposition}{命題}[section]
\newtheorem{problem}{問題}[section]
\newtheorem{example}{例}[section]


\def\arithmean#1{\bar{#1}}
\def\arithsum#1{#1}
\def\harmmean#1{\hat{#1}}
\def\harmformula#1#2#3{\frac{#1}{\frac{1}{#2} + \frac{1}{#3}}}

\title{調和平均}

\begin{document}

\maketitle

\section{調和平均の性質}

\begin{definition}[調和平均]
正の実数 $x_1, x_2, \ldots, x_n$,
($x_i \in \mathbb{R}, x_i > 0$) について
\begin{equation}
H = \frac{n}{\frac{1}{x_1} + \frac{1}{x_2} + \cdots \frac{1}{x_n}}
= \frac{n}{\sum_{i=1}^{n} \frac{1}{x_i}}
\end{equation}
を調和平均 (harmonic mean) と呼ぶ。
\end{definition}

\begin{definition}[相加平均]
正の実数 $x_1, x_2, \ldots, x_n$,
($x_i \in \mathbb{R}, x_i > 0$) について
\begin{equation}
A = \frac{x_1 + x_2 + \cdots x_n}{n}
= \frac{\sum_{i=1}^{n} x_i}{n}
\end{equation}
を相加平均 (または算術平均, arithmetic mean) と呼ぶ。
\end{definition}

\begin{definition}[相乗平均]
正の実数 $x_1, x_2, \ldots, x_n$,
($x_i \in \mathbb{R}, x_i > 0$) について
\begin{equation}
G = {}^{n} \sqrt{x_1 x_2 \cdots x_n}
\end{equation}
を相乗平均 (または幾何平均, geomtric mean) と呼ぶ。
\end{definition}

\begin{theorem}[平均の不等式]
\begin{equation}
H \le G \le A
\end{equation}
が成り立つ。
\end{theorem}

\begin{definition}[記号]
$n$ 個の正の実数の組
\begin{align}
(a_i, b_i), && a_i \in \mathbb{R}, b_i \in \mathbb{R},
a_i > 0, b_i >0 \qquad i = 1, 2, \ldots n
\end{align}
に対して、
\begin{align}
h_i = & \harmformula{2}{a_i}{b_i},
& {h_i}' = & \harmformula{1}{a_i}{b_i}
\end{align}
とおく。さらに、
\begin{align}
\arithsum{A} = & \sum_{i=1}^{n} a_i,
& \arithmean{a} = &
\frac{1}{n} \sum_{i=1}^{n} a_i
= \frac{1}{n} A, \\
\arithsum{B} = & \sum_{i=1}^{n} b_i,
& \arithmean{b} = &
\frac{1}{n} \sum_{i=1}^{n} b_i
= \frac{1}{n} B, \\
\arithsum{H} = & \sum_{i=1}^{n} {h_i}',
& \arithmean{h} = &
\frac{1}{n} \sum_{i=1}^{n} h_i
= \frac{2}{n} H, \\
\harmmean{H} = & \harmformula{1}{A}{B},
& \harmmean{h} = & \harmformula{2}{\arithmean{a}}{\arithmean{b}}
\end{align}
と記号を定義する。
\end{definition}

\begin{lemma}
$\forall x, y, s, t > 0 (\in \mathbb{R})$ に対して
\begin{equation}
\harmformula{1}{x}{y} + \harmformula{1}{s}{t}
\le \harmformula{1}{x + s}{y + t}
\end{equation}
が成り立つ。
\end{lemma}

\begin{theorem}
調和平均と相加平均に関して、
\begin{equation}
\arithmean{h} \le \harmmean{h}
\end{equation}
が成り立つ。i.e.,
ペア毎に調和平均をとってから相加平均を取ったものは、
先に $a$ 列 $b$ 列で相加平均を取ってから調和平均を取ったものより
大きくはない。
\end{theorem}


\end{document}
